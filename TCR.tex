\documentclass[a4paper,11pt]{article}

\usepackage[a4paper, margin=0.8in]{geometry}

\usepackage[english]{babel}
\usepackage[utf8]{inputenc}
\usepackage[T1]{fontenc}
\usepackage{amsmath,amssymb,amsfonts,amsthm,dsfont,commath}
\usepackage{graphicx}
\usepackage{listings}
\usepackage[usenames,dvipsnames,table]{xcolor}
\usepackage[pdftex,colorlinks=true]{hyperref}
\usepackage{fancyhdr}
\usepackage{multicol}

\hypersetup{
  bookmarksnumbered,
  linkcolor=RoyalBlue,
  anchorcolor=RoyalBlue,
  citecolor=RoyalBlue,
  urlcolor=RoyalBlue,
  pdfstartview={FitV},
  pdfdisplaydoctitle
}

\definecolor{shade}{RGB}{245,245,245}

\lstset{
  tabsize=2,
  numbers=left,
  breaklines=true,
  framexleftmargin=0.05in,
  basicstyle=\ttfamily\small,
  numberstyle=\tiny,
  backgroundcolor=\color{shade},
  keywordstyle=\color{RoyalBlue},
  stringstyle=\color{Maroon},
  commentstyle=\color{ForestGreen},
  language=C++,
  %multicols=2,
}

\pagestyle{fancy}
\lhead{University of Copenhagen}
\rhead{\thepage}
\cfoot{}
\renewcommand{\headrulewidth}{0.4pt}



\title{Team Reference\\
       University of Copenhagen}
\author{Ballondyrene}
\date{\today}

\begin{document}
\maketitle
\thispagestyle{fancy}
\tableofcontents

\section{Introduction}
  \subsection{Preamble}
\begin{lstlisting}
#include <cstdlib>
#include <bits/stdc++.h>
using namespace std;
#define ll long long
#define com complex<double>
#define ld long double;
#define ii pair<int,int>
#define vi vector<int>
#define vii vector<ii>
#define vc vector<char>
#define vs vector<string>
#define vd vector<double>
#define vcom vector<com>
#define vld vector<ld>
#define vll vector<ll>
#define vvi vector<vi>
#define vvii vector<vii>
#define vvc vector<vc>
#define vvs vector<vs>
#define vvll vector<vll>
#define vvd vector<vd>
#define vvcom vector<vcom>
#define vvld vector<vld>
#define FOR(x,n) for(int x=0;x<(n);x++)
#define FORS(x,n) for(int x=1;x<=(n);x++)
#define FORE(x,a) for(auto &x: a)
#define FORT(x,a,b) for(int x=(a);x<(b);x++)
#define FORD(x,a,b) for(int x=(a);x>=(b);x--)
#define ALL(x) x.begin(),x.end()
#define REP(n) for(int _ = 0; _ < n; _++)
#define MT make_tuple
#define MP make_pair
#define pb push_back
#define endl '\n'
#define F first
#define S second

void solve() {
}

int main() {
    ios_base::sync_with_stdio(false);
    cin.tie(NULL);
    int t = 1;
    //cin >> t;
    while(t--) {
        solve();
    }
}
\end{lstlisting}
  
  \subsection{Commonly used functions}
\begin{lstlisting}
int approxTwo(int n) {
  n = n - 1;
  n |= n >> 1;
  n |= n >> 2;
  n |= n >> 4;
  n |= n >> 8;
  n |= n >> 16;
  return n + 1;
}
\end{lstlisting}

\section{Data Structures}
  \subsection{Fenwick Tree}
    \lstinputlisting[firstline=9,lastline=22]{Data-structures/Fenwick-tree/fenwick.cpp}
    Two dimensional
    \lstinputlisting[firstline=10,lastline=27]{Data-structures/Fenwick-tree/two-dimensional.cpp}
  
  \subsection{Segment Tree}
    \lstinputlisting[firstline=16,lastline=89]{Data-structures/segment-tree/Segment-tree.cpp}    
    
  \subsection{Treap}
    \lstinputlisting[firstline=6,lastline=53]{Data-structures/Treap/treap-new.cpp}
  
  \subsection{Disjoint Set}
    \lstinputlisting[firstline=9,lastline=26]{Data-structures/Disjoint-set/disjoint-set.cpp}    

\section{Graph}
  \subsection{Minimum Spanning Tree}
    \lstinputlisting[firstline=11,lastline=55]{Graph/Minimum-spanning-tree/kruskal.cpp}

  \subsection{Single Source Shortest Path}
    If there are no negative edges use Dijkstra algorithm
    \lstinputlisting[firstline=13,lastline=41]{Graph/Single-source-shortest-path/dijkstra.cpp}    
    
    If there are negative edges use Bellman-Ford algorithm
    \lstinputlisting[firstline=13,lastline=43]{Graph/Single-source-shortest-path/bellman-ford.cpp}

  \subsection{Multiple Sources Shortest Path}
    \lstinputlisting[firstline=7,lastline=19]{Graph/Multiple-sources-shortest-path/floyd-warshall.cpp}
  
  \subsection{Maximum Flow}
    \lstinputlisting[firstline=9,lastline=58]{Graph/Max-flow/Scaling.cpp}
  
  \subsection{Minimum Cost Maximum Flow}
    \lstinputlisting[firstline=13,lastline=110]{Graph/Min-cost-max-flow/SSP.cpp}
  	\lstinputlisting[firstline=13,lastline=87]{Graph/Min-cost-max-flow/Levitic_MCMF.cpp}
  \subsection{Maximum Matching on Bipartite Graph}
    \lstinputlisting[firstline=12,lastline=77]{Graph/Max-matching/bipartite.cpp}
  
  \subsection{Strongly Connected Components}
    \lstinputlisting[firstline=10,lastline=67]{Graph/Strongly-connected-components/SCC.cpp}

\section{String}
  \subsection{Z-algorithm}
    \lstinputlisting[firstline=28,lastline=44]{String/z-algorithm/Z.cpp}  

  \subsection{Suffix automaton}
    \lstinputlisting[firstline=9,lastline=50]{String/suffix/suffix.cpp}  
  
  \subsection{Longest Common Subsequence}
    \lstinputlisting[firstline=11,lastline=40]{String/longest-common-subsequence/LCS.cpp}

\section{Number Theory}
  \lstinputlisting[firstline=13,lastline=65]{Number-theory/number-theory.cpp}

\section{Geometry}
  \lstinputlisting[firstline=11,lastline=226]{Geometry/geometry2.cpp}

\section{Miscellaneous}
  \subsection{Longest Increasing Subsequence}
    \lstinputlisting[firstline=5,lastline=27]{Miscellaneous/Longest-increasing-subsequence/LIS2.cpp}    
    
  \subsection{Fast Fourier Transform}
    \lstinputlisting[firstline=13,lastline=67]{Miscellaneous/FFT/FFT.cpp}

  \subsection{Minimum weight matching}
    \lstinputlisting[firstline=11,lastline=61]{Miscellaneous/Min-weight-matching/hungarian.cpp}

\end{document}
